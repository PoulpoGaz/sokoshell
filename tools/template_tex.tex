\documentclass[
    convert={density=300, outext=.png},
    command=\unexpanded{-quality 90}
]{standalone}

\usepackage{tikz}
\usetikzlibrary{babel}
\usetikzlibrary{shapes, positioning, calc}
\usetikzlibrary{math}

\usepackage{graphics}

% French language support (e.g. date format)
\usepackage[french]{babel}
\usepackage[T1]{fontenc}
\usepackage{lmodern} % for missing fonts (e.g. italic in titles)

\makeatletter

\newif\iftikz@sok@grid
\tikz@sok@gridfalse

\pgfkeys{
    /tikz/.cd,
    sok/img width/.initial = -1cm,
    sok/x/.initial = 0,
    sok/y/.initial = 0,
    sok/width/.initial = 1,
    sok/height/.initial = 1,
    sok/grid/.code={\tikz@sok@gridtrue},
}

% https://tex.stackexchange.com/questions/27836/saving-computed-number-registers-for-access-outside-body-of-let-operation
\pgfkeys{/tikz/savenumber/.code 2 args={\global\edef#1{#2}}}

\newenvironment{sokoban}[2][]{
    \tikzset{#1}% read key values

    \pgfmathsetlengthmacro{\imageWidth}{\pgfkeysvalueof{/tikz/sok/img width}};
    \pgfmathsetlengthmacro{\x}         {\pgfkeysvalueof{/tikz/sok/x}};
    \pgfmathsetlengthmacro{\y}         {\pgfkeysvalueof{/tikz/sok/y}};
    \pgfmathsetmacro{\width}           {\pgfkeysvalueof{/tikz/sok/width}};
    \pgfmathsetmacro{\height}          {\pgfkeysvalueof{/tikz/sok/height}};

    % draw image with specified width
    \node[anchor=north west, inner sep=0] (image) at (\x, \y) {
        % https://tex.stackexchange.com/questions/531866/ifnum-of-variable
        \pgfmathparse{\imageWidth<0}
        \ifnum\pgfmathresult=1
        \includegraphics{#2}
        \else
        \includegraphics[width=\imageWidth]{#2}
        \fi
    };

    \path
        let \p{east} = (image.east),
            \p{west} = (image.west),
    \n{width} = {\x{east}-\x{west}}
    in
    [savenumber={\imageWidth}{\n{width}}];

    \begin{scope}[xshift = \x,
        yshift = \y,
        x = \imageWidth / \width,
        y = -\imageWidth / \width]
        \iftikz@sok@grid
        \draw[help lines, xstep=1, ystep=-1, color=blue] (0, 0) grid (\width, \height);
        \fi
        }{
    \end{scope}
}

\makeatother

\newcommand{\fsokoban}[3]{
    \begin{sokoban}[sok/width=#2, sok/height=#3]{#1}

    \end{sokoban}
}
