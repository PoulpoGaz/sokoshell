\documentclass[]{beamer}

\usetheme[subsectionpage=progressbar]{metropolis}

\makeatletter
\setlength{\metropolis@titleseparator@linewidth}{3pt}
\setlength{\metropolis@progressonsectionpage@linewidth}{3pt}
\setlength{\metropolis@progressinheadfoot@linewidth}{3pt}
\makeatother

\usepackage{tikz}
\usetikzlibrary{babel}
\usetikzlibrary{arrows,shapes,positioning,shadows,trees,calc,fit}
\usetikzlibrary{overlay-beamer-styles} % 'visible on' option for nodes
\usepackage{adjustbox}


\title{Résolution de niveaux du Sokoban}
\date{\today}
\author{PoulpoGaz, darth-mole}
\institute{Candidat n° 012345}

\usepackage{graphics}
\graphicspath{{../../assets/}}
\usepackage{subcaption}

% French language support (e.g. date format)
\usepackage[french]{babel}
\usepackage[T1]{fontenc}
\usepackage{lmodern} % for missing fonts (e.g. italic in titles)

\newenvironment{customtree}[1][0.8]{
    \begin{adjustbox}{max totalsize={\textwidth}{\textheight}, center}
        \begin{tikzpicture}
            [sibling distance = 10em,
            level distance = 7em,
            every node/.style = {
                % minimum width=10em,
                shape=rectangle,
                % line width=0.4mm,
                scale=#1,
                inner sep=0pt,
                outer sep=0pt
            },
            edge from parent/.style = {
                draw,
                edge from parent path = {
                    (\tikzparentnode) |-                          % Start from parent
                    ($(\tikzparentnode)!0.5!(\tikzchildnode)$) -| % make an ortho line to mid point
                    (\tikzchildnode)                              % make another ortho to the target
                },
                line width=0.4mm
            },
            % root/.style = {
            % },
        ]
}{
        \end{tikzpicture}
    \end{adjustbox}
}

\newcommand\iconwidth{2em}
\newcommand\arrowwidth{0.8mm}